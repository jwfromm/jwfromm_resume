%%%%%%%%%%%%%%%%%%%%%%%%%%%%%%%%%%%%%%%%%%%%%%%%%%%%%%%%%%%%%%%%%%%%%%%%
%%%%%%%%%%%%%%%%%%%%%% Simple LaTeX CV Template %%%%%%%%%%%%%%%%%%%%%%%%
%%%%%%%%%%%%%%%%%%%%%%%%%%%%%%%%%%%%%%%%%%%%%%%%%%%%%%%%%%%%%%%%%%%%%%%%

%%%%%%%%%%%%%%%%%%%%%%%%%%%%%%%%%%%%%%%%%%%%%%%%%%%%%%%%%%%%%%%%%%%%%%%%
%% NOTE: If you find that it says                                     %%
%%                                                                    %%
%%                           1 of ??                                  %%
%%                                                                    %%
%% at the bottom of your first page, this means that the AUX file     %%
%% was not available when you ran LaTeX on this source. Simply RERUN  %%
%% LaTeX to get the ``??'' replaced with the number of the last page  %%
%% of the document. The AUX file will be generated on the first run   %%
%% of LaTeX and used on the second run to fill in all of the          %%
%% references.                                                        %%
%%%%%%%%%%%%%%%%%%%%%%%%%%%%%%%%%%%%%%%%%%%%%%%%%%%%%%%%%%%%%%%%%%%%%%%%

%%%%%%%%%%%%%%%%%%%%%%%%%%%% Document Setup %%%%%%%%%%%%%%%%%%%%%%%%%%%%

% Don't like 10pt? Try 11pt or 12pt
\documentclass[10pt]{article}

% This is a helpful package that puts math inside length specifications
\usepackage{calc}


% Simpler bibsection for CV sections
% (thanks to natbib for inspiration)
\makeatletter
\newlength{\bibhang}
\setlength{\bibhang}{1em}
\newlength{\bibsep}
 {\@listi \global\bibsep\itemsep \global\advance\bibsep by\parsep}
\newenvironment{bibsection}%
        {\vspace{-\baselineskip}\begin{list}{}{%
       \setlength{\leftmargin}{\bibhang}%
       \setlength{\itemindent}{-\leftmargin}%
       \setlength{\itemsep}{\bibsep}%
       \setlength{\parsep}{\z@}%
        \setlength{\partopsep}{0pt}%
        \setlength{\topsep}{0pt}}}
        {\end{list}\vspace{-.6\baselineskip}}
\makeatother

% Layout: Puts the section titles on left side of page
\reversemarginpar

%
%         PAPER SIZE, PAGE NUMBER, AND DOCUMENT LAYOUT NOTES:
%
% The next \usepackage line changes the layout for CV style section
% headings as marginal notes. It also sets up the paper size as either
% letter or A4. By default, letter was used. If A4 paper is desired,
% comment out the letterpaper lines and uncomment the a4paper lines.
%
% As you can see, the margin widths and section title widths can be
% easily adjusted.
%
% ALSO: Notice that the includefoot option can be commented OUT in order
% to put the PAGE NUMBER *IN* the bottom margin. This will make the
% effective text area larger.
%
% IF YOU WISH TO REMOVE THE ``of LASTPAGE'' next to each page number,
% see the note about the +LP and -LP lines below. Comment out the +LP
% and uncomment the -LP.
%
% IF YOU WISH TO REMOVE PAGE NUMBERS, be sure that the includefoot line
% is uncommented and ALSO uncomment the \pagestyle{empty} a few lines
% below.
%

%% Use these lines for letter-sized paper
\usepackage[paper=letterpaper,
            %includefoot, % Uncomment to put page number above margin
            marginparwidth=1.2in,     % Length of section titles
            marginparsep=.05in,       % Space between titles and text
            margin=0.7in,               % 1 inch margins
            includemp]{geometry}

%% Use these lines for A4-sized paper
%\usepackage[paper=a4paper,
%            %includefoot, % Uncomment to put page number above margin
%            marginparwidth=30.5mm,    % Length of section titles
%            marginparsep=1.5mm,       % Space between titles and text
%            margin=25mm,              % 25mm margins
%            includemp]{geometry}

%% More layout: Get rid of indenting throughout entire document
\setlength{\parindent}{0in}

%% This gives us fun enumeration environments. compactitem will be nice.
\usepackage{paralist}

%% Reference the last page in the page number
%
% NOTE: comment the +LP line and uncomment the -LP line to have page
%       numbers without the ``of ##'' last page reference)
%
% NOTE: uncomment the \pagestyle{empty} line to get rid of all page
%       numbers (make sure includefoot is commented out above)
%
\usepackage{fancyhdr,lastpage}
\pagestyle{fancy}
\pagestyle{empty}      % Uncomment this to get rid of page numbers
\fancyhf{}\renewcommand{\headrulewidth}{0pt}
\fancyfootoffset{\marginparsep+\marginparwidth}
\newlength{\footpageshift}
\setlength{\footpageshift}
          {0.5\textwidth+0.5\marginparsep+0.5\marginparwidth-2in}
\lfoot{\hspace{\footpageshift}%
       \parbox{4in}{\, \hfill %
                    \arabic{page} of \protect\pageref*{LastPage} % +LP
%                    \arabic{page}                               % -LP
                    \hfill \,}}

% Finally, give us PDF bookmarks
\usepackage{color,hyperref}
\definecolor{darkblue}{rgb}{0.0,0.0,0.3}
\hypersetup{colorlinks,breaklinks,
            linkcolor=darkblue,urlcolor=darkblue,
            anchorcolor=darkblue,citecolor=darkblue}

%%%%%%%%%%%%%%%%%%%%%%%% End Document Setup %%%%%%%%%%%%%%%%%%%%%%%%%%%%


%%%%%%%%%%%%%%%%%%%%%%%%%%% Helper Commands %%%%%%%%%%%%%%%%%%%%%%%%%%%%

% The title (name) with a horizontal rule under it
% (optional argument typesets an object right-justified across from name
%  as well)
%
% Usage: \makeheading{name}
%        OR
%        \makeheading[right_object]{name}
%
% Place at top of document. It should be the first thing.
% If ``right_object'' is provided in the square-braced optional
% argument, it will be right justified on the same line as ``name'' at
% the top of the CV. For example:
%
%       \makeheading[\emph{Curriculum vitae}]{Your Name}
%
% will put an emphasized ``Curriculum vitae'' at the top of the document
% as a title. Likewise, a picture could be included:
%
%   \makeheading[\includegraphics[height=1.5in]{my_picutre}]{Your Name}
%
% the picture will be flush right across from the name.
\newcommand{\makeheading}[2][]%
        {\hspace*{-\marginparsep minus \marginparwidth}%
         \begin{minipage}[t]{\textwidth+\marginparwidth+\marginparsep}%
             {\large \bfseries #2 \hfill #1}\\[-0.15\baselineskip]%
                 \rule{\columnwidth}{1pt}%
         \end{minipage}}

% The section headings
%
% Usage: \section{section name}
%
% Follow this section IMMEDIATELY with the first line of the section
% text. Do not put whitespace in between. That is, do this:
%
%       \section{My Information}
%       Here is my information.
%
% and NOT this:
%
%       \section{My Information}
%
%       Here is my information.
%
% Otherwise the top of the section header will not line up with the top
% of the section. Of course, using a single comment character (%) on
% empty lines allows for the function of the first example with the
% readability of the second example.
\renewcommand{\section}[2]%
        {\pagebreak[3]\vspace{1.3\baselineskip}%
         \phantomsection\addcontentsline{toc}{section}{#1}%
         \hspace{0in}%
         \marginpar{
         \raggedright \scshape #1}#2}

% An itemize-style list with lots of space between items
\newenvironment{outerlist}[1][\enskip\textbullet]%
        {\begin{itemize}[#1]}{\end{itemize}%
         \vspace{-.6\baselineskip}}

% An environment IDENTICAL to outerlist that has better pre-list spacing
% when used as the first thing in a \section
\newenvironment{lonelist}[1][\enskip\textbullet]%
        {\vspace{-\baselineskip}\begin{list}{#1}{%
        \setlength{\partopsep}{0pt}%
        \setlength{\topsep}{0pt}}}
        {\end{list}\vspace{-.6\baselineskip}}

% An itemize-style list with little space between items
\newenvironment{innerlist}[1][\enskip\textbullet]%
        {\begin{compactitem}[#1]}{\end{compactitem}}

% An environment IDENTICAL to innerlist that has better pre-list spacing
% when used as the first thing in a \section
\newenvironment{loneinnerlist}[1][\enskip\textbullet]%
        {\vspace{-\baselineskip}\begin{compactitem}[#1]}
        {\end{compactitem}\vspace{-.6\baselineskip}}

% To add some paragraph space between lines.
% This also tells LaTeX to preferably break a page on one of these gaps
% if there is a needed pagebreak nearby.
\newcommand{\blankline}{\quad\pagebreak[3]}
\newcommand{\halfblankline}{\quad\vspace{-0.5\baselineskip}\pagebreak[3]}

% Uses hyperref to link DOI
\newcommand\doilink[1]{\href{http://dx.doi.org/#1}{#1}}
\newcommand\doi[1]{doi:\doilink{#1}}

% For \url{SOME_URL}, links SOME_URL to the url SOME_URL
\providecommand*\url[1]{\href{#1}{#1}}
% Same as above, but pretty-prints SOME_URL in teletype fixed-width font
\renewcommand*\url[1]{\href{#1}{\texttt{#1}}}

% For \email{ADDRESS}, links ADDRESS to the url mailto:ADDRESS
\providecommand*\email[1]{\href{mailto:#1}{#1}}
% Same as above, but pretty-prints ADDRESS in teletype fixed-width font
%\renewcommand*\email[1]{\href{mailto:#1}{\texttt{#1}}}

%\providecommand\BibTeX{{\rm B\kern-.05em{\sc i\kern-.025em b}\kern-.08em
%    T\kern-.1667em\lower.7ex\hbox{E}\kern-.125emX}}
%\providecommand\BibTeX{{\rm B\kern-.05em{\sc i\kern-.025em b}\kern-.08em
%    \TeX}}
\providecommand\BibTeX{{B\kern-.05em{\sc i\kern-.025em b}\kern-.08em
    \TeX}}
\providecommand\Matlab{\textsc{Matlab}}

%%%%%%%%%%%%%%%%%%%%%%%% End Helper Commands %%%%%%%%%%%%%%%%%%%%%%%%%%%

%%%%%%%%%%%%%%%%%%%%%%%%% Begin CV Document %%%%%%%%%%%%%%%%%%%%%%%%%%%%
\begin{document}
\makeheading{Josh Fromm}

\section{Contact Information}
%
% NOTE: Mind where the & separators and \\ breaks are in the following
%       table.
%
% ALSO: \rcollength is the width of the right column of the table
%       (adjust it to your liking; default is 1.85in).
%
\newlength{\rcollength}\setlength{\rcollength}{2.5in}%
%
\begin{tabular}[t]{@{}p{\textwidth-\rcollength}p{\rcollength}}
2510B NE 65th St	& \textit{Mobile:} +1-626-676-2684 \\
Seattle, WA 98115 & \textit{E-mail:} \email{jwfromm@uw.edu}\\ &\textit{Website: }\href{http://www.jwfromm.com}{jwfromm.com}\\
\end{tabular}

\section{Research Statement}
I am a fifth year PhD Student at the University of Washington advised by Shwetak Patel in the \textbf{Ubiquitous Computing Lab}.
My research focuses on (1) Developing \textbf{novel approximating algorithms} for neural networks,
(2) building systems that enable \textbf{efficient execution on embedded platforms},
and (3) exploring applications of \textbf{computer vision in resource constrained environments}.

\section{Education}
%
\href{http://www.uw.edu/}{\textbf{University of Washington}}, Seattle, WA
\begin{innerlist}
\item[] Pursuing a Ph.D in Electrical Engineering as part of the UbiComp Lab.\hfill\textbf{2014 - 2019}.
\end{innerlist}
\href{http://www.caltech.edu/}{\textbf{California Institute of Technology}}, Pasadena, CA
\begin{innerlist}
\item[] Bachelor of Science with Honors in Electrical Engineering \hfill \textbf{June 2014} \\ and Computer Science.
\end{innerlist}

\section{Experience}
%
\textbf{\href{http://ubicomplab.cs.washington.edu/}{Graduate Student}}\hfill \textbf{2014 to present}
	\begin{innerlist}
		\item[] \href{https://ubicomplab.cs.washington.edu/}{UbiComp Lab} \hfill \textbf{Research Assistant}
	\begin{innerlist}
	        \item[] \emph{Researching novel applications and architectures for deep neural networks, with an emphasis on high performance computer vision systems.}
	    \end{innerlist}
	\end{innerlist}
	
\textbf{\href{https://nest.com/}{Google Nest}}\hfill \textbf{2018}
	\begin{innerlist}
		\item[] \href{https://ai.google/research/teams/brain}{Nest Brain Team} \hfill \textbf{Research Intern}
	\begin{innerlist}
	        \item[] \emph{Developed novel generative adversarial network training techniques to enable realistic colorization of IR images captured by Nest Cams.}
	    \end{innerlist}
	\end{innerlist}

\textbf{\href{http://research.microsoft.com/}{Microsoft Research}}\hfill \textbf{2016 and 2017}
	\begin{innerlist}
		\item[] \href{https://www.microsoft.com/en-us/research/project/machine-learning-edge/}{Machine Learning on the Edge Group} \hfill \textbf{Research Intern}
	\begin{innerlist}
	        \item[] \emph{Explored neural network binarization as a method for enabling deep convolutional neural network inference on Raspberry Pi class devices. Developed novel binarization algorithms to allow fine-grained tuning of speed and accuracy tradeoff.}
	    \end{innerlist}
	\end{innerlist}

\textbf{\href{http://research.microsoft.com/}{Microsoft Research Cambridge}}\hfill \textbf{2015}
	\begin{innerlist}
		\item[] \href{http://research.microsoft.com/en-us/groups/sendev/default.aspx}{Sensors and Devices Team} \hfill \textbf{Research Intern}
	\begin{innerlist}
            \item[] \emph{Developed RF power harvesting techniques and hardware as part of the NEXT initiative to create novel interaction technology.}
	    \end{innerlist}
	\end{innerlist}
	
\textbf{\href{http://www.nvidia.com/}{Nvidia Corporation}}\hfill \textbf{2013 and 2014}
	\begin{innerlist}
		\item[] \href{http://www.geforce.com/hardware}{GPU Verification Division} \hfill \textbf{ASIC Engineer}
	\begin{innerlist}
	        \item[] \emph{Verified that streaming multiprocessor operation in RTL matched simulated outputs using a C++ model.}
	    \end{innerlist}
	\end{innerlist}

\textbf{\href{http://www.jpl.nasa.gov/}{NASA Jet Propulsion Laboratory}}
	\begin{innerlist}
		\item[] \href{https://www-robotics.jpl.nasa.gov/}{Chris Assad Lab, Robotics Division} \hfill \hfill \textbf{SURF Fellow 2012}
	    \begin{innerlist}
        \item[] \emph{Designed and developed EMG electrode system that allows control of a robotic arm through muscle activity.}
	    \end{innerlist}
	\end{innerlist}
% Add a little space to nudge next ``Conference Publications'' marginpar
% down to make room for tall ``Submitted Journal Publications''
% marginpar. If there are enough submitted journal publications, this
% space will not be needed (and should be removed).
%\vspace{0.1in}

\section{Conference\\Publications} \begin{bibsection}
    \item \href{https://arxiv.org/abs/1805.10368}{Fromm J, Patel S, Phillipose M. Heterogeneous Bitwidth Binarization in Convolutional Neural Networks. In: NeurIPS, 2018.}
    \item {Moreau T, Chen T, Fromm J, et al. YOGI: Flexible Architecture \& Runtime Co-Design for Deep Learning Specialization. In: ISCA 2018.}
    \item {Saba E, Fromm J, Jiayao C, Patel S. TB or not TB: Cough Detection and Tuberculosis Classification for Pulmonary Health Estimation. In: IMWUT, 2018.}
    \item \href{https://ieeexplore.ieee.org/document/8462127/}{Hwan Ko J, Fromm J, Phillipose M, Tashev I, Zarar S. Liming Numerical Precision of Neural Networks to Achieve Real-Time Voice Activity Detection. In: ICASSP, 2018.}
    \item \href{http://www.jwfromm.com/documents/paperid.pdf}{Li H, Brockmeyer E, Carter E, Fromm J, Hudson S, Patel S, Sample A. PaperID: A Technique for Drawing Functional Battery-Free Wireless Interfaces on Paper. In: CHI, 2016.}
    \item \href{http://www.grosse-puppendahl.com/publications/uist2016.pdf}{Grosse-Puppendahl T et al. Exploring the Design Space for Energy-Harvesting Situated Displays. In: UIST 2016.}
    \item \href{http://www.jwfromm.com/documents/spirocall.pdf}{Goel M, Saba E, Stiber M, Whitmire E, Fromm J, Larson E, Borriello G, Patel S. SpiroCall: Measuring Lung Function over a Phone Call. In: CHI, 2016.}
    \item \href{http://jwfromm.com/documents/BioSleeve_ICRA.pdf}{Wolf M, Assad C, Vernacchia M, Fromm J, Jethani H. Gesture-Based Robot Control with Variable Autonomy from the JPL BioSleeve. In: IEEE Conference on Robotics and Automation (ICRA), 2013.}
\end{bibsection}

\section{Honors and Awards}
    Microsoft Research Graduate Fellow \hfill \textbf{2017}\\
    NSF Graduate Research Fellowship Honorable Mention \hfill \textbf{2016}\\
    Google IOT Research Award Recipient \hfill \textbf{2016}\\
    Amazon Catalyst Fellow \hfill \textbf{2016}\\
    Qualcomm Innovation Fellowship Finalist \hfill \textbf{2015}\\
    Caltech Bachelors of Science with Honors \hfill \textbf{2014}\\
    Caltech Uppper Class Merit Award \hfill \textbf{2013}\\
    Richter Scholar Fellow \hfill \textbf{2011}\\
    Lincoln Southeast High School Valedictorian \hfill \textbf{2010}

\section{Teaching Experience}
\href{https://myplan.uw.edu/course/#/courses/TECHIN513}{\textbf{GIX TECHIN 513: Managing Data and Signal Processing}} \hfill \textbf{Instructor 2017-2019}
    \begin{innerlist}
    \item[] Developed and taught an introduction to practical deep learning with a data and application driven focus. The course teaches students how to create an end to end deep
            learning system with their own data and deploy it efficiently on custom hardware or in the cloud.
    \end{innerlist}

\textbf{UW EE 478: Embedded Systems Capstone} \hfill \textbf{Instructor 2015}
    \begin{innerlist}
    \item[] Developed and taught a course for seneior embedded design students meant to emulate an industry experience.
            In the course, teams of students propsed, designed, and built a custom embedded system from scratch and had
            to deliver on predtermined milestones.
    \end{innerlist}

\textbf{UW EE 472: Embedded Microcomputer Systems} \hfill \textbf{TA 2015}
    \begin{innerlist}
    \item[] Redesigned the curriculum of an intro to embedded systems course to focus on creating
            a cohesive "RoboTank" mobile robotic platform from scratch over the quarter.
    \end{innerlist}

\textbf{Caltech EE/CS 51: Embedded Systems Software Design Laboratory} \hfill \textbf{TA 2012-2014}
    \begin{innerlist}
    \item[] Intro to embedded systems that focuses on developing efficient firmware in assembly.
            Primary skills developed in this course are careful planning, system design, and most importantly debugging.
    \end{innerlist}

\textbf{Caltech EE/CS 52: Embedded Systems Software Design Laboratory} \hfill \textbf{TA 2012-2014}
    \begin{innerlist}
    \item[] Embedded systems hardware course in which students develop a voice over IP phone from scratch using
            both custom hardware and software (assembly).
    \end{innerlist}

\textbf{Caltech EE/CS 53: Microprocessor Project Laboratory} \hfill \textbf{TA 2013-2014}
    \begin{innerlist}
    \item[] Advanced embedded systems course in which students propose and develop a project of their choosing.
    \end{innerlist}

\section{References}
\begin{tabular}[t]{@{}p{\textwidth-\rcollength}p{\rcollength}}
    Shwetak Patel                              & Steve Hodges \\
    Professor and Director of the Ubicomp Lab  & Senior Researcher \\
    Computer Science \& Electrical Engineering & Sensors and Devices Team \\
    University of Washington                   & Microsoft Research Cambridge \\
    shwetak@cs.washington.edu                  & steve.hodges@microsoft.com \\

    \\
    Luis Ceze                                  & Matthai Philipose \\
    Professor and Director of the SAMPA Lab    & Senior Researcher \\
    Computer Science                           & Mobility and Networking Group \\
    University of Washington                   & Microsoft Research \\
    luisceze@cs.washington.edu                 & matthaip@microsoft.com \\
\end{tabular}

\end{document}

%%%%%%%%%%%%%%%%%%%%%%%%%% End CV Document %%%%%%%%%%%%%%%%%%%%%%%%%%%%%

%----------------------------------------------------------------------%
% The following is copyright and licensing information for
% redistribution of this LaTeX source code; it also includes a liability
% statement. If this source code is not being redistributed to others,
% it may be omitted. It has no effect on the function of the above code.
%----------------------------------------------------------------------%
% Copyright (c) 2007, 2008, 2009, 2010, 2011 by Theodore P. Pavlic
%
% Unless otherwise expressly stated, this work is licensed under the
% Creative Commons Attribution-Noncommercial 3.0 United States License. To
% view a copy of this license, visit
% http://creativecommons.org/licenses/by-nc/3.0/us/ or send a letter to
% Creative Commons, 171 Second Street, Suite 300, San Francisco,
% California, 94105, USA.
%
% THE SOFTWARE IS PROVIDED "AS IS", WITHOUT WARRANTY OF ANY KIND, EXPRESS
% OR IMPLIED, INCLUDING BUT NOT LIMITED TO THE WARRANTIES OF
% MERCHANTABILITY, FITNESS FOR A PARTICULAR PURPOSE AND NONINFRINGEMENT.
% IN NO EVENT SHALL THE AUTHORS OR COPYRIGHT HOLDERS BE LIABLE FOR ANY
% CLAIM, DAMAGES OR OTHER LIABILITY, WHETHER IN AN ACTION OF CONTRACT,
% TORT OR OTHERWISE, ARISING FROM, OUT OF OR IN CONNECTION WITH THE
% SOFTWARE OR THE USE OR OTHER DEALINGS IN THE SOFTWARE.
%----------------------------------------------------------------------%
